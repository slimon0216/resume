\documentclass{resume} % Use the custom resume.cls style

\usepackage[left=0.4in,top=0.4in,right=0.4in, bottom=0.4in]{geometry} % Document margins
\newcommand{\tab}[1]{\hspace{.2667\textwidth}\rlap{#1}} 
\newcommand{\itab}[1]{\hspace{0em}\rlap{#1}}


\name{Po-Yu Chen} % Your name
% You can merge both of these into a single line, if you do not have a website.
\address{+(886) 983-960-657 \\ \href{mailto:slimon0216@gmail.com}{slimon0216@gmail.com} \\ \href{https://github.com/slimon110}{GitHub} \\ \href{https://www.linkedin.com/in/po-yu-chen-999277227} {linkedin}}  %

\begin{document}

%----------------------------------------------------------------------------------------
%	OBJECTIVE
%----------------------------------------------------------------------------------------

\begin{rSection}{OBJECTIVE}
% Software Engineer with 2+ years of experience in XXX, seeking full-time XXX roles.
Seeking a software engineering position where I can leverage my skills in programming, computer systems, machine learning and problem solving to make valuable contributions to the team
\end{rSection}

%----------------------------------------------------------------------------------------
%	EDUCATION SECTION
%----------------------------------------------------------------------------------------

\begin{rSection}{Education}

{\bf National Taiwan University (NTU)} \hfill \textbf{Taipei, Taiwan}
\begin{itemize}
    % \itemsep -3pt
    \item {M.S.} in {Electrical Engineering}, GPA: 4.20/4.30, Class Rank: 2/24  \hfill \textit{Sep. 2022 - Jun. 2024(Expected)}
    \item {B.B.A.} in {International Business}, GPA: 3.82/4.30 \hfill \textit{Sep. 2017 - Jun. 2022}
    \item {B.B.A.} in {Information Management} (Double Major)  \hfill \textit{Sep. 2017 - Jun. 2022}
    \item {Relevant Coursework}: Operating System, Computer Networks/Architectures, Machine Learning, Computer Vision 
\end{itemize}

\end{rSection}


% ----------------------------------------------------------------------------------------
% TECHINICAL STRENGTHS	
%----------------------------------------------------------------------------------------

\begin{rSection}{SKILLS}
    \begin{tabular}{ @{} >{\bfseries}l @{\hspace{6ex}} l }
    Technical &  C/C++, CMake, Git, Unix/Linux, Bash, Python, PyTorch, GNU toolchain, Docker, CUDA \\ 
    Language & Mandarin (Native), English (Intermediate), Japanese (Beginner)
    \end{tabular}\\
\end{rSection}



%----------------------------------------------------------------------------------------
%	WORK EXPERIENCE SECTION
%----------------------------------------------------------------------------------------

\begin{rSection}{WORK EXPERIENCE} 

\textbf{MeidaTek} \hfill \textbf{Hsinchu, Taiwan} \\
\textit{Software Engineer Intern} \hfill {July 2023 - Sep 2023 (Expected)} 
\begin{itemize}
    \item Developed a parser to transform CPU profiler and tracer outputs into Google's protobuf format, enabling in-depth visualization and analysis using Google's Perfetto Trace Viewer
\end{itemize}

\textbf{Awesome Research} \hfill \textbf{Taipei, Taiwan} \\  
\textit{Software Engineer and Quantitative Trader}  \hfill \textit{Jan 2022 - Sep 2022} 
    \begin{itemize}
        % \itemsep -3pt {} 
        \item Led a 5-person infra team to develop C++ toolkit to support trading and research teams 
        \item Implemented an automated arbitrage strategy in C++ independently, with well-designed error and signal handling mechanism, resulting in consistent profits for the company 
        \item Wrote a script to automate the build and testing process, resulting in a more efficient and streamlined development workflow
        \item Reduced the system build time by over 50\% by adopting a compiler cache to avoid unnecessary recompilations
        \item Increased test coverage from 40\% to 90\% by writing unit tests using tools like GoogleTest
        \item Refactored tightly-coupled codes and eliminated duplications to enhance the reusability and maintainability
    \end{itemize}




% \textbf{National Taiwan University} \hfill \textit{Sep 2021 - Jan 2022} \\
% \textit{Teaching Assistant, Programming for Business Computing, Prof. Ling-Chieh Kung} \hfill \textit{Taipei, Taiwan}
% \begin{itemize}
%     \item Checked and corrected the contents and test data of weekly homework before released
%     \item Held weekly office hour to answer questions for 100+ students
% \end{itemize}
    
\end{rSection} 


\begin{rSection}{PROJECTS}
\vspace{-1.25em}

% \item {\bf Matrix Multiplication Optimization for Graph Neural Network(GNN)} \hfill \textit{Feb. 2023 - Present}\\
% Integrated custom C++/CUDA functions into python as extensions through PyTorch C++/CUDA api.

\item {\bf 3D Indoor Scene Semantic Segmentation} \hfill \textit{Nov, 2022 -  Jan, 2023}
\begin{itemize}
    \item Leveraged a pre-trained language model and contrastive loss to guide the U-Net backbone in learning 3D features
\end{itemize}

\item {\bf Byzantine Distributed Optimization on Non-linear Constrained Problem} (\href{https://github.com/slimon110/2022_Fault_Tolerance/blob/master/FT-Report/FT-Report.pdf}{Paper with Code})  \hfill \textit{Oct, 2022 - Jan, 2023}
\begin{itemize}
    \itemsep -3pt
    \item Collaborated with lab colleagues, implemented the proposed algorithms and conducted experiments with python
\end{itemize}

\item \textbf{Trademark Retrieval Mobile App} (\href{https://play.google.com/store/apps/details?id=meow.logoshot&hl=zh_TW&gl=TW&pli=1}{Google Play}/\href{https://apps.apple.com/tw/app/logo-shot/id1611756574}{App Store}) \hfill \textit{Feb. 2021 - Jan. 2022}
\begin{itemize}
    \itemsep -3pt
    \item  Built a CNN-based model to learn image representations for downstream image retrieval task 
\end{itemize}
% \end{itemize}

\item \textbf{Secure P2P Micro-payment System} \hfill \textit{Nov. 2021 - Jan. 2022} 
\begin{itemize}
    \itemsep -3pt
    \item Built client and server that communicated through secure TCP/SSL socket by using C++ multithreading in a producer-consumer scheme
    % \item Used a job queue and thread pool to enable the server to handle multiple connections in parallel, and used mutex, condition variable to guarantee thread-safety and non-busy waiting.
    % \item Leverage to enable the server to handle multiple connections in parallel, and used mutex, condition variable to guarantee thread-safety and non-busy waiting.
\end{itemize}

\item \textbf{Product Recommendation System for Fubon Financial Holding Co., Ltd.} \hfill \textit{Feb. 2021 - Jun. 2021}
\begin{itemize}
    \itemsep -3pt
    \item Used ensemble and over-and-under-sampling techniques to build a classification model with high AUC score that equally matched with the real inner system adopted by Fubon
\end{itemize}

% \item \textbf{Taipei Pre-owned Housing Prices Prediction} \hfill \textit{Sep. 2020 - Jan. 2021} \\
% Built a price-prediction model to alleviate the price non-transparency problem in secondhand house market. Used Google Geocoding API to collect the distance from each house to nearest MRT station to enrich data and improve performance.

% \item \textbf{House price}
% \item \textbf{capstone}


\end{rSection} 

%----------------------------------------------------------------------------------------
\begin{rSection}{Extra-Curricular Activities} 
\textbf{National Taiwan University Guitar Club}, \textit{Education Officier} \hfill \textit{Sep. 2020 - Jun. 2021}
% \begin{itemize}
%     \item Instructed weekly guitar classes for a hundreds of club members
% \end{itemize}

\textbf{Taipei Medical University 36th Guitar Competition}, \textit{Finalist}  \hfill \textit{Nov. 2019 - Dec. 2019} 


\textbf{National Taiwan University Orientation Camp}, \textit{Camp counselor}  \hfill \textit{Jun. 2019 - Aug. 2019} 


% \textbf{orientation camp}

\end{rSection}

%----------------------------------------------------------------------------------------



\end{document}
